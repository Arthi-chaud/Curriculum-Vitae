%-------------------------
% Resume in Latex
% Author : Vaishanth
% License : MIT
%------------------------

\begin{document}
%----------HEADING----------


\begin{center}
	{\huge Arthur Jamet} \\ \vspace{0.5cm}
	\small{}
	\href{https://arthichaud.xyz}{\color{blue}{arthichaud.xyz}} ~
	\small{-}
	\href{mailto:arthur.jamet@gmail.com}{\color{blue}{arthur.jamet@gmail.com}} ~
	\small{-}
	\href{https://linkedin.com/in/arthur-jamet}{\color{blue}{linkedin.com/in/arthur-jamet}} ~
	\small{-}
	\href{https://github.com/Arthi-chaud}{\color{blue}{github.com/Arthi-chaud}} ~
	\vspace{0.1cm}
\end{center}

%-----------EDUCATION-----------
\section{\color{airforceblue}
  \en{EDUCATION}
  \fr{FORMATIONS}
 }
\resumeSubHeadingListStart
\resumeSubheading
{
	\en{University of Kent}
	\fr{Université du Kent}
}{
	\en{Canterbury, England}
	\fr{Canterbury, Angleterre}
}
{
	\en{PhD Student - Supervised by Michael Vollmer}
	\fr{Doctorant - Supervised by Michael Vollmer}
}{
	\en{September 2024 - Present}
	\fr{Septembre 2024 - Présent}
}

\hspace{\parindent}
\en{Research Subject: Enhancing programs' performance by changing the layout of data in memory}
\fr{Sujet de recherche: Optimisation de logiciels en changeant la représentation des données dans la mémoire vive}

\resumeSubheading
{
	\en{EPITECH}
	\fr{EPITECH}
}
{Nantes, France}
{
	\en{Master in Information Technologies}
	\fr{Master en Technologies de l'Information}
}
{
	\en{September 2019 - September 2024}
	\fr{Septembre 2019 - Septembre 2024}
}

\hspace{\parindent} \hspace{\parindent} GPA: 4.00/4.00
\resumeSubheading
{
	\en{University of Kent}
	\fr{Université du Kent}
}{
	\en{Canterbury, England}
	\fr{Canterbury, Angleterre}
}
{
	\en{MSc Advanced Computed Science (With Distinctions)}
	\fr{Master en Sciences Avancées de l'Informatique (Avec Mention Très Bien)}
}{
	\en{September 2022 - September 2023}
	\fr{Septembre 2022 - Septembre 2023}
}

\hspace{\parindent}
\en{Dissertation: Generating Better Type-Error messages for GHC (Glasgow Haskell Compiler)}
\fr{Sujet de mémoire: Générer de meilleurs messages d'erreurs liées aux types pour GHC (Glasgow Haskell Compiler)}

\resumeSubHeadingListEnd
\vspace{-10pt}

%-----------PROGRAMMING SKILLS-----------
\section{\color{airforceblue}
  \en{TECHNICAL SKILLS}
  \fr{SKILLS TECHNIQUES}
 }
\begin{itemize}[leftmargin=0in, label={}]
	\small{\item{
	      \textbf{\normalsize{
			      \en{Programming Languages:}
			      \fr{Langages de programmation:}
		      }}{\normalsize{TypeScript/JavaScript, Haskell, C, Rust, Dart, Python, PHP, C++}} \\
	      \vspace{2.4pt}
	      \textbf{\normalsize{
			      \en{Programming Frameworks:}
			      \fr{Frameworks:}
		      }}{\normalsize{Nest, Servant, Symfony, Rocket / React / React-Native, Flutter}} \\
	      \vspace{2.4pt}

	      \textbf{\normalsize{
			      \en{Tools:}
			      \fr{Outils:}
		      }}{\normalsize{Bash, Git, GitHub, Docker, LaTeX, LSP, OAuth}} \\
	      \vspace{2.4pt}
	      }}
\end{itemize}
\vspace{-16pt}

%-----------EXPERIENCE-----------
\section{\color{airforceblue}
  \en{WORK EXPERIENCE}
  \fr{EXPÉRIENCES}
 }
\resumeSubHeadingListStart

\resumeSubheading
{
	\en{Graduate Teaching Assistant}
	\fr{Assitant Pédagogique}
}{}
{
	\en{University of Kent, Canterbury, England}
	\fr{Université du Kent, Canterbury, Angleterre}
}
{
	\en{September 2024 - Present}
	\fr{Septembre 2024 - Présent}
}
\resumeItemListStart
\resumeItem{\normalsize{
		\en{Seminar and class supervisor, for Object-Oriented and Functional Programming modules.}
		\fr{En charge des séminaires et travaux pratiques, pour les modules de programmation fonctionnelle et orientée-objet.}
	}}
\resumeItemListEnd
\resumeSubheading
{
	\en{Full-Stack Developer}
	\fr{Développeur Full-Stack}
}{}
{
	\en{Lucca, Nantes, France}
	\fr{Lucca, Nantes, France}
}
{
	\en{March - September 2024}
	\fr{Mars - Septembre 2024}
}
\resumeItemListStart
\resumeItem{\normalsize{
		\en{Internship. Integration of Slack and Teams Notifications with the 'Poplee Engagement' software.}
		\fr{Stage. Intégration de notifications Slack et Teams au logiciel 'Poplee Engagement'.
		}}}
\resumeItemListEnd

\resumeSubheading
{
	\en{Teaching Assistant}
	\fr{Assitant Pédagogique}
}{}
{
	\en{EPITECH, Nantes, France}
	\fr{EPITECH, Nantes, France}
}
{
	\en{January 2021 - July 2022 / September 2023 - February 2024}
	\fr{Janvier 2021 - Juillet 2022 / Septembre 2023 - Février 2024}
}
\resumeItemListStart
\resumeItem{\normalsize{
		\en{Support and evaluate students during their curriculum.}
		\fr{Accompagnement et évaluation des étudiants pendant leur cursus.}
	}}
\resumeItem{\normalsize{
		\en{Assist teachers in managing resources, enhancing pedagogical content and improve the school's life quality.}
		\fr{Aide à la gestion des resources (salles, effectifs), à l'amélioration du matériel pédagogique, et participation à la vie de l'école.}
	}}
\resumeItemListEnd

\resumeSubheading
{
	\en{Web Developer}
	\fr{Développeur Web}
}{}
{
	\en{Union Mutualiste Retraite, Nantes, France}
	\fr{Union Mutualiste Retraite, Nantes, France}
}
{
	\en{September 2020 - December 2020}
	\fr{Septembre 2020 - Décembre 2020}
}
\resumeItemListStart
\resumeItem{\normalsize{
		\en{Internship. Develop and deploy a web app to monitor internal file streams}
		\fr{Stage. Développement et déploiement d'une application web permettant de monitorer les flux internes de fichiers}
	}}
\resumeItemListEnd
% copy and paste above resumeSubheading block to add more 
\resumeSubHeadingListEnd
\vspace{-12pt}

%-----------Publications---------------
\section{\color{airforceblue}PUBLICATIONS}

\resumeItemListStart
\resumeItem{\normalsize{\textit{Type-safe and portable support for packed data} - \textbf{A. Jamet}, M.Vollmer (ECOOP 2025)}} \href{https://drops.dagstuhl.de/entities/document/10.4230/LIPIcs.ECOOP.2025.38}{\color{blue}\underline{Link}}
% \vspace{-5pt}



\resumeItemListEnd

\vspace{-10pt}

%-----------PROJECTS-----------
\section{\color{airforceblue}
  \en{PROJECTS}
  \fr{PROJETS}
 }
\resumeItemListStart
\vspace{0.5pt}
\resumeItem{\normalsize{
		\textbf{Meelo},
		\en{Music Server, designed for collectors. Personal project. Written with Nest (Server), Next (Web app).}
		\fr{Serveur de musique, désigné pour les collectionneurs. Projet personnel. Réalisé avec Nest (Server) et Next (App Web).}
	}
	\href{https://github.com/Arthi-chaud/Meelo}{\color{blue}\underline{github.com/Arthi-chaud/Meelo}}}
\vspace{-5pt}
% \resumeItem{\normalsize{
% 		\textbf{AERIS},
% 		\en{Pipelines manager (like Zapier). School Project (3rd Year). Written in Haskell (Server), TypeScript (Worker), React (Web app) and Flutter (Mobile app).}
% 		\fr{Gestionnaire de pipelines (à la Zapier). Projet d'école (3rd Year). Écrit en Haskell (API), TypeScript (Worker), React (App Web) and Flutter (App Mobile).}
% 	}
% 	\href{[url to project github repo]}{\color{blue}\underline{github.com/Zoriya/aeris}}}
% \vspace{-5pt}
% \resumeItem{\normalsize{
% 		\textbf{ChromaCase},
% 		\en{Web and Mobile to help you learning the piano, using traditional music sheets. End-of-studies project, written in TypeScript (Server, Web/Mobile App) and Python (Scrapper, Web-Socket Microservice).}
% 		\fr{Application Web et Mobile pour aider à l'apprentissage du piano, en utilisant les partitions traditionnelles. Projet de fin d'études, réalisé en TypeScript (API, App Web et Mobile) et Python (Scrapper et Micro-service utilisant des Web-Sockets)}
% 	}
% 	\href{[url to project github repo]}{\color{blue}\underline{github.com/Chroma-Case/Chromacase}}}
% \vspace{-5pt}

\resumeItem{\normalsize{
		\textbf{Lambdananas Language Server},
		\en{Language Server for Lambdananas, EPITECH's Coding Style Checker for Haskell. Written in Haskell. Includes an LSP client for VSCode (written in TypeScript).}
		\fr{Language Server pour Lambdananas, scanneur de norme pour Haskell d'EPITECH. Ecrit en Haskell. Inclus un client LSP pour VSCode (écrit en TypeScript). }
	}
	\href{https://github.com/Arthi-chaud/lambdananas-language-server}{\color{blue}\underline{github.com/Arthi-chaud/lambdananas-language-server}}}
\resumeItemListEnd

\vspace{-12pt}
%


%-----------EXTRACURRICULAR---------------
% \section{\color{airforceblue}EXTRACURRICULAR ACTIVITIES}

%       \resumeItemListStart
%         \resumeItem{\normalsize{\textbf{Event Coordinator} in univesrity Club - {Month Year - Month Year} }} 
%         \vspace{-5pt}

%         \resumeItem{\normalsize{\textbf{Executive member} in Robotics Club - } {Month Year - Month Year}}    
%         \vspace{-5pt}

%         % copy and paste above resumeItem block to add more 
%       \resumeItemListEnd 

% \vspace{-12pt}

\end{document}
